\documentclass[11pt,reqno]{amsart}
\usepackage{amssymb,mathrsfs,color}
\usepackage{pinlabel}
\usepackage{graphicx}
\usepackage{graphics} 

\graphicspath{ {c:/users/mcytrynbaum/documents/rfigures/} {c:/users/mcytrynbaum/Desktop/package/} }
\DeclareGraphicsExtensions{.pdf,.png,.jpg}

%\usepackage[notcite,notref]{showkeys}

%\textwidth=15cm \textheight=21.5cm
%\oddsidemargin=0.5cm \evensidemargin=0.5cm
\usepackage{amsmath} % for all math functions and operations
\usepackage{amsfonts} % use this to write different scripts (e.g. Real nums, etc)
\usepackage{mathtools} %for other math stuff not included in packages above
\usepackage{amsthm} % in case you want the THM: COR: LEMMA: setup
\usepackage[top=1in,bottom=1in,left=1in,right=1in]{geometry} %for setting the margins

\setlength\parindent{0pt}

\newtheorem{thm}{Theorem}[section]
\newtheorem{lemma}[thm]{Lemma}
\newtheorem{prop}[thm]{Proposition}
\newtheorem{cor}[thm]{Corollary}
\theoremstyle{definition}
\newtheorem{defn}[thm]{Definition}
\newtheorem{examp}[thm]{Example}
\newtheorem{remark}[thm]{Remark}
\setcounter{equation}{0}
\numberwithin{equation}{section}

\newcommand{\prf}{\begin{proof}}
\newcommand{\eprf}{\end{proof}}
\newcommand{\lft}{\left(}
\newcommand{\rt}{\right)}
\newcommand{\be}{\beta}
\newcommand{\eps}{\epsilon}
\newcommand{\wh}{\widehat}
\newcommand{\wt}{\widetilde}
\newcommand{\al}{\alpha}
\newcommand{\bp}{\begin{pmatrix}}
\newcommand{\ep}{\end{pmatrix}}
\newcommand{\inv}{^{-1}}
\newcommand{\var}{\text{Var}}
\newcommand{\cov}{\text{Cov}}
\newcommand{\corr}{\text{Corr}}
\newcommand{\ssumi}{\sum_{i=1}^n}
\newcommand{\ssumj}{\sum_{j=1}^n}
\newcommand{\ssumk}{\sum_{k=1}^n}
\newcommand{\im}{\text{Im}}
\newcommand{\mc}{\mathcal}
\newcommand{\mr}{\mathbb{R}}
\newcommand{\ol}{\overline}
\newcommand{\ul}{\underline}
\newcommand{\prob}{\mathbb{P}}
\newcommand{\ital}{\emph}
\newcommand{\tb}{\textbf}
\newcommand{\pa}{\partial}
\newcommand{\dt}{\delta}
\newcommand{\tc}{t_{comm}}
\newcommand{\tp}{t_{patent}}
\newcommand{\et}{\eta}
\newcommand{\ov}{\overline}

\title{Towards Density of SOSX in SMSX}
\author{Max Cytrynbaum, Wei Hu}

\begin{document}
\maketitle

I record some attempted solutions and possible avenues for proving density $SOSX \longrightarrow SMSX$. \\

\tb{Convolution Approach} - One idea was to use the original idea from the Weierstrass Approximation Theorem (convolution with a polynomial kernel) to guarantee convexity. Convolution (over $\mr^p$) with a positive kernel preserves convexity, while convolution with a polynomial kernel (over a compact set) yields a polynomial. The issue is that we cannot guarantee both of these simultaneously. \\

Our convex function $f$ is defined on a compact set $K$. Shift and extend $f$ to $K_1 \supset K$ such that $f = 0$ on $K_1^c$. Let $g(x)$ be a standard normal density in $\mr^p$. It follows from a theorem in Prolla (1988) that the cone of psd polynomials is dense (in the uniform norm) in the cone of continuous psd functions on $K$. Then let $p_m \to g$ be such a sequence of psd polynomials. \\

The idea is to notice that 

\[
\int_{\mr^p} f(x - t)p_m(t) dt = \int_{-K + x} f(x - t)p_m(t) dt = \int_{K} f(t)p_m(x - t)
\]

Notice that the last expression is clearly a polynomial in x. We want to say that the first integral is convex (as a convolution of a convex function with a psd kernel $p_m$. We would actually want to build $p_m^{\eps} \to \frac{1}{\eps} g(x / \eps)$ uniformly on some set containing $K$, following the standard convolution argument. \\

In the context of the truncated convolution above, we get uniform convergence on any compact set contained in $K$. Extending to $K \subset \subset K_1$ gives uniform convergence on $K$.  \\

The issue is that we need to do something to make the integral converge. Making $f = 0$ on $K_1^c$ destroys convexity of the first expression, while if we trucate the polynomial outside of $K$, the last expression will no longer be a polynomial. An argument of this type seemed promising at first, but it is not clear if we can actually get something like this to work. \\

\tb{Gradient Approximation Approach} - last time, we took an $SMSX$ function $f$ with Hessian $H_f = L_f L_f^T$ and approximated with $L_f^m \to L_f$ in $\| \cdot \|_{\infty}$. This was supposed to give an approximation $p_m \to f$ such that all $p_m$ are $SOSX$ polynomials. The issue is that our approximation $H_f^m$ is not necessarily a Hessian. I can show that given a gradient $g = \nabla h$, under some regularity conditions there exists a polynomial gradient $\nabla p_m \to g$ uniformly on a compact set in the case $p = 2$.\\

The proof takes advantage of the fact that the ``curl-free'' conditions are very nice in 2-dimensions. It is not obvious how to extend this argument to the general case. 





\end{document}